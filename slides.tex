%!TEX program = xelatex
\documentclass[12pt, compress, t]{beamer}
%% ODER: format ==         = "\mathrel{==}"
%% ODER: format /=         = "\neq "
%
%
\makeatletter
\@ifundefined{lhs2tex.lhs2tex.sty.read}%
  {\@namedef{lhs2tex.lhs2tex.sty.read}{}%
   \newcommand\SkipToFmtEnd{}%
   \newcommand\EndFmtInput{}%
   \long\def\SkipToFmtEnd#1\EndFmtInput{}%
  }\SkipToFmtEnd

\newcommand\ReadOnlyOnce[1]{\@ifundefined{#1}{\@namedef{#1}{}}\SkipToFmtEnd}
\usepackage{amstext}
\usepackage{amssymb}
\usepackage{stmaryrd}
\DeclareFontFamily{OT1}{cmtex}{}
\DeclareFontShape{OT1}{cmtex}{m}{n}
  {<5><6><7><8>cmtex8
   <9>cmtex9
   <10><10.95><12><14.4><17.28><20.74><24.88>cmtex10}{}
\DeclareFontShape{OT1}{cmtex}{m}{it}
  {<-> ssub * cmtt/m/it}{}
\newcommand{\texfamily}{\fontfamily{cmtex}\selectfont}
\DeclareFontShape{OT1}{cmtt}{bx}{n}
  {<5><6><7><8>cmtt8
   <9>cmbtt9
   <10><10.95><12><14.4><17.28><20.74><24.88>cmbtt10}{}
\DeclareFontShape{OT1}{cmtex}{bx}{n}
  {<-> ssub * cmtt/bx/n}{}
\newcommand{\tex}[1]{\text{\texfamily#1}}	% NEU

\newcommand{\Sp}{\hskip.33334em\relax}


\newcommand{\Conid}[1]{\mathit{#1}}
\newcommand{\Varid}[1]{\mathit{#1}}
\newcommand{\anonymous}{\kern0.06em \vbox{\hrule\@width.5em}}
\newcommand{\plus}{\mathbin{+\!\!\!+}}
\newcommand{\bind}{\mathbin{>\!\!\!>\mkern-6.7mu=}}
\newcommand{\rbind}{\mathbin{=\mkern-6.7mu<\!\!\!<}}% suggested by Neil Mitchell
\newcommand{\sequ}{\mathbin{>\!\!\!>}}
\renewcommand{\leq}{\leqslant}
\renewcommand{\geq}{\geqslant}
\usepackage{polytable}

%mathindent has to be defined
\@ifundefined{mathindent}%
  {\newdimen\mathindent\mathindent\leftmargini}%
  {}%

\def\resethooks{%
  \global\let\SaveRestoreHook\empty
  \global\let\ColumnHook\empty}
\newcommand*{\savecolumns}[1][default]%
  {\g@addto@macro\SaveRestoreHook{\savecolumns[#1]}}
\newcommand*{\restorecolumns}[1][default]%
  {\g@addto@macro\SaveRestoreHook{\restorecolumns[#1]}}
\newcommand*{\aligncolumn}[2]%
  {\g@addto@macro\ColumnHook{\column{#1}{#2}}}

\resethooks

\newcommand{\onelinecommentchars}{\quad-{}- }
\newcommand{\commentbeginchars}{\enskip\{-}
\newcommand{\commentendchars}{-\}\enskip}

\newcommand{\visiblecomments}{%
  \let\onelinecomment=\onelinecommentchars
  \let\commentbegin=\commentbeginchars
  \let\commentend=\commentendchars}

\newcommand{\invisiblecomments}{%
  \let\onelinecomment=\empty
  \let\commentbegin=\empty
  \let\commentend=\empty}

\visiblecomments

\newlength{\blanklineskip}
\setlength{\blanklineskip}{0.66084ex}

\newcommand{\hsindent}[1]{\quad}% default is fixed indentation
\let\hspre\empty
\let\hspost\empty
\newcommand{\NB}{\textbf{NB}}
\newcommand{\Todo}[1]{$\langle$\textbf{To do:}~#1$\rangle$}

\EndFmtInput
\makeatother
%
%
%
%
%
%
% This package provides two environments suitable to take the place
% of hscode, called "plainhscode" and "arrayhscode". 
%
% The plain environment surrounds each code block by vertical space,
% and it uses \abovedisplayskip and \belowdisplayskip to get spacing
% similar to formulas. Note that if these dimensions are changed,
% the spacing around displayed math formulas changes as well.
% All code is indented using \leftskip.
%
% Changed 19.08.2004 to reflect changes in colorcode. Should work with
% CodeGroup.sty.
%
\ReadOnlyOnce{polycode.fmt}%
\makeatletter

\newcommand{\hsnewpar}[1]%
  {{\parskip=0pt\parindent=0pt\par\vskip #1\noindent}}

% can be used, for instance, to redefine the code size, by setting the
% command to \small or something alike
\newcommand{\hscodestyle}{}

% The command \sethscode can be used to switch the code formatting
% behaviour by mapping the hscode environment in the subst directive
% to a new LaTeX environment.

\newcommand{\sethscode}[1]%
  {\expandafter\let\expandafter\hscode\csname #1\endcsname
   \expandafter\let\expandafter\endhscode\csname end#1\endcsname}

% "compatibility" mode restores the non-polycode.fmt layout.

\newenvironment{compathscode}%
  {\par\noindent
   \advance\leftskip\mathindent
   \hscodestyle
   \let\\=\@normalcr
   \let\hspre\(\let\hspost\)%
   \pboxed}%
  {\endpboxed\)%
   \par\noindent
   \ignorespacesafterend}

\newcommand{\compaths}{\sethscode{compathscode}}

% "plain" mode is the proposed default.
% It should now work with \centering.
% This required some changes. The old version
% is still available for reference as oldplainhscode.

\newenvironment{plainhscode}%
  {\hsnewpar\abovedisplayskip
   \advance\leftskip\mathindent
   \hscodestyle
   \let\hspre\(\let\hspost\)%
   \pboxed}%
  {\endpboxed%
   \hsnewpar\belowdisplayskip
   \ignorespacesafterend}

\newenvironment{oldplainhscode}%
  {\hsnewpar\abovedisplayskip
   \advance\leftskip\mathindent
   \hscodestyle
   \let\\=\@normalcr
   \(\pboxed}%
  {\endpboxed\)%
   \hsnewpar\belowdisplayskip
   \ignorespacesafterend}

% Here, we make plainhscode the default environment.

\newcommand{\plainhs}{\sethscode{plainhscode}}
\newcommand{\oldplainhs}{\sethscode{oldplainhscode}}
\plainhs

% The arrayhscode is like plain, but makes use of polytable's
% parray environment which disallows page breaks in code blocks.

\newenvironment{arrayhscode}%
  {\hsnewpar\abovedisplayskip
   \advance\leftskip\mathindent
   \hscodestyle
   \let\\=\@normalcr
   \(\parray}%
  {\endparray\)%
   \hsnewpar\belowdisplayskip
   \ignorespacesafterend}

\newcommand{\arrayhs}{\sethscode{arrayhscode}}

% The mathhscode environment also makes use of polytable's parray 
% environment. It is supposed to be used only inside math mode 
% (I used it to typeset the type rules in my thesis).

\newenvironment{mathhscode}%
  {\parray}{\endparray}

\newcommand{\mathhs}{\sethscode{mathhscode}}

% texths is similar to mathhs, but works in text mode.

\newenvironment{texthscode}%
  {\(\parray}{\endparray\)}

\newcommand{\texths}{\sethscode{texthscode}}

% The framed environment places code in a framed box.

\def\codeframewidth{\arrayrulewidth}
\RequirePackage{calc}

\newenvironment{framedhscode}%
  {\parskip=\abovedisplayskip\par\noindent
   \hscodestyle
   \arrayrulewidth=\codeframewidth
   \tabular{@{}|p{\linewidth-2\arraycolsep-2\arrayrulewidth-2pt}|@{}}%
   \hline\framedhslinecorrect\\{-1.5ex}%
   \let\endoflinesave=\\
   \let\\=\@normalcr
   \(\pboxed}%
  {\endpboxed\)%
   \framedhslinecorrect\endoflinesave{.5ex}\hline
   \endtabular
   \parskip=\belowdisplayskip\par\noindent
   \ignorespacesafterend}

\newcommand{\framedhslinecorrect}[2]%
  {#1[#2]}

\newcommand{\framedhs}{\sethscode{framedhscode}}

% The inlinehscode environment is an experimental environment
% that can be used to typeset displayed code inline.

\newenvironment{inlinehscode}%
  {\(\def\column##1##2{}%
   \let\>\undefined\let\<\undefined\let\\\undefined
   \newcommand\>[1][]{}\newcommand\<[1][]{}\newcommand\\[1][]{}%
   \def\fromto##1##2##3{##3}%
   \def\nextline{}}{\) }%

\newcommand{\inlinehs}{\sethscode{inlinehscode}}

% The joincode environment is a separate environment that
% can be used to surround and thereby connect multiple code
% blocks.

\newenvironment{joincode}%
  {\let\orighscode=\hscode
   \let\origendhscode=\endhscode
   \def\endhscode{\def\hscode{\endgroup\def\@currenvir{hscode}\\}\begingroup}
   %\let\SaveRestoreHook=\empty
   %\let\ColumnHook=\empty
   %\let\resethooks=\empty
   \orighscode\def\hscode{\endgroup\def\@currenvir{hscode}}}%
  {\origendhscode
   \global\let\hscode=\orighscode
   \global\let\endhscode=\origendhscode}%

\makeatother
\EndFmtInput
%
\usetheme[titleprogressbar]{m}
\usepackage{tikz-cd}       % Коммутативные диаграммы
\usepackage{booktabs}
\usepackage[scale=2]{ccicons}
\usepackage{minted}
\usepackage{listings}

\usepgfplotslibrary{dateplot}

% \usemintedstyle{trac}

% \lstset{language=Haskell,
%         basicstyle=\footnotesize}

% Большие номера слайдов формата i/N
\setbeamerfont{page number in head/foot}{size=\large}
\setbeamertemplate{footline}[frame number]
% Большие номера слайдов формата i/N

% Команда для удобной вставки скриншота
\newcommand{\screenshotw}[2]{
  \centering\includegraphics[width=#1,keepaspectratio]{./images/#2}
}
% Команда для удобной вставки скриншота


% Форматирование <|>

\definecolor{cloud}{HTML}{ECF0F1}
\setbeamercolor{block title}{fg=white,bg=orange!75!black}
\setbeamercolor{block body}{fg=black,bg=cloud}

\title{Функциональный парсер легковесного языка разметки Markdown
на основе комбинирования монад и моноидального представления исходного текста}
\date{20.06.2015}
\author{Г.~А.~Лукьянов, ПМИ, группа 4.1 \\{Научный руководитель: асс. каф. ИВЭ А.~М.~Пеленицын}}
\institute{Институт математики, механики и компьютерных наук ЮФУ}

\renewcommand{\hscodestyle}{\small}

\begin{document}

\maketitle

\begin{frame}[fragile]
  \frametitle{Поставленные задачи}
  \begin{enumerate}
    \item Разработка двух библиотек \textbf{монадических парсеров}, 
    с применением различных технологий \textbf{комбинирования вычислительных 
    эффектов}. Обе библиотеки должны использовать \textbf{моноидальное}
    представление исходного текста.
    \item \textbf{Сравнение подходов} к комбинированию вычислительных эффектов, выявление их преимуществ и недостатков.
    \item Разработка \textbf{транслятора Markdown} с \LaTeX-вставками в HTML и \LaTeX.
  \end{enumerate}
\end{frame}

\begin{frame}[fragile]
  \frametitle{Использовались результаты работ: }
  \begin{description}
    \item [MParsers96]
    Monadic Parser Combinators // \textit{Graham Hutton}, \textit{Erik Meijer} –
Department of Computer Science, University of Nottingham, 1996
    \item [Monoids13]
    Adding Structure to Monoids // \textit{Mario Blaževic} – Stilo International plc – \\\textbf{Haskell Symposium 2013}
    \item [ExtEff13] 
    Extensible Effects An Alternative to Monad Transformers // \textit{Oleg
Kiselyov}, \textit{Amr Sabry}, \textit{Cameron Swords} – Indiana University, USA – \textbf{Haskell Symposium 2013}
  \end{description}
\end{frame}

\begin{frame}[fragile]
  \frametitle{Библиотеки монадических парсеров}
  \begin{itemize}
    \setlength\itemsep{2em}
    \item[] {\Large{Parsec}} \\
      \footnotesize{\url{https://hackage.haskell.org/package/parsec}}
    \item[] {\Large{Attoparsec}} \\
      \footnotesize{\url{https://hackage.haskell.org/package/attoparsec}}
  \end{itemize}
\end{frame}

\begin{frame}[fragile]
  \frametitle{Исходный тип для парсера~[MParsers96]}
  \begin{block}{Тип Parser}
    \begin{hscode}\SaveRestoreHook
\column{B}{@{}>{\hspre}l<{\hspost}@{}}%
\column{10}{@{}>{\hspre}l<{\hspost}@{}}%
\column{E}{@{}>{\hspre}l<{\hspost}@{}}%
\>[B]{}\mathbf{newtype}\;{}\<[10]%
\>[10]{}\Conid{Parser}\;\Varid{a}\mathrel{=}\Conid{Parser}\;\{\mskip1.5mu {}\<[E]%
\\
\>[10]{}\Varid{parse}\mathbin{::}\Conid{String}\to \Conid{Maybe}\;(\Varid{a},\Conid{String})\mskip1.5mu\}{}\<[E]%
\ColumnHook
\end{hscode}\resethooks
  \end{block}
  \begin{block}{Экземпляр класса типов Monad}
    \vspace{-1em}
  	\begin{hscode}\SaveRestoreHook
\column{B}{@{}>{\hspre}l<{\hspost}@{}}%
\column{11}{@{}>{\hspre}l<{\hspost}@{}}%
\column{20}{@{}>{\hspre}l<{\hspost}@{}}%
\column{25}{@{}>{\hspre}l<{\hspost}@{}}%
\column{29}{@{}>{\hspre}l<{\hspost}@{}}%
\column{37}{@{}>{\hspre}l<{\hspost}@{}}%
\column{E}{@{}>{\hspre}l<{\hspost}@{}}%
\>[B]{}\mathbf{instance}\;{}\<[11]%
\>[11]{}\Conid{Monad}\;\Conid{Parser}\;\mathbf{where}{}\<[E]%
\\
\>[11]{}\Varid{return}\;\Varid{t}\mathrel{=}\Conid{Parser}\mathbin{\$}\lambda \Varid{s}\to \Conid{Just}\;(\Varid{t},\Varid{s}){}\<[E]%
\\
\>[11]{}\Varid{m}\bind \Varid{k}{}\<[20]%
\>[20]{}\mathrel{=}\Conid{Parser}\mathbin{\$}\lambda \Varid{s}\to {}\<[E]%
\\
\>[20]{}\hsindent{5}{}\<[25]%
\>[25]{}\mathbf{do}\;{}\<[29]%
\>[29]{}(\Varid{u},\Varid{v}){}\<[37]%
\>[37]{}\leftarrow \Varid{parse}\;\Varid{m}\;\Varid{s}{}\<[E]%
\\
\>[29]{}(\Varid{x},\Varid{y}){}\<[37]%
\>[37]{}\leftarrow \Varid{parse}\;(\Varid{k}\;\Varid{u})\;\Varid{v}{}\<[E]%
\\
\>[29]{}\Varid{return}\;(\Varid{x},\Varid{y}){}\<[E]%
\ColumnHook
\end{hscode}\resethooks
  \end{block}
\end{frame}

\begin{frame}[fragile]
  \frametitle{Строковые типы в Haskell}
  \begin{itemize}
    \setlength\itemsep{2em}
    \item[] {\Large{String}} \\
      \small{Псевдоним для списка символов}
    \item[] {\Large{ByteString}} \\
      \small{Наиболее низкоуровневый тип}
    \item[] {\Large{Text}} \\
      \small{Тип для работы с Unicode-текстом}
  \end{itemize}
\end{frame}

\begin{frame}[fragile]
  \frametitle{Строковые типы как моноиды~[Monoids]}
    \begin{block}{Полиморфный по входу базовый парсер}
      \begin{hscode}\SaveRestoreHook
\column{B}{@{}>{\hspre}l<{\hspost}@{}}%
\column{9}{@{}>{\hspre}l<{\hspost}@{}}%
\column{E}{@{}>{\hspre}l<{\hspost}@{}}%
\>[B]{}\Varid{item}\mathbin{::}\Conid{TextualMonoid}\;\Varid{t}\Rightarrow \Conid{Parser}\;\Varid{t}\;\Conid{Char}{}\<[E]%
\\
\>[B]{}\Varid{item}\mathrel{=}{}\<[9]%
\>[9]{}\Conid{Parser}\;\Varid{f}{}\<[E]%
\\
\>[9]{}\mathbf{where}\;\Varid{f}\;\Varid{inp}\mathrel{=}\Varid{splitCharacterPrefix}\;\Varid{inp}{}\<[E]%
\ColumnHook
\end{hscode}\resethooks
    \end{block}
    \begin{block}{Фукнция, отделяющая префикс}
      \begin{hscode}\SaveRestoreHook
\column{B}{@{}>{\hspre}l<{\hspost}@{}}%
\column{26}{@{}>{\hspre}l<{\hspost}@{}}%
\column{E}{@{}>{\hspre}l<{\hspost}@{}}%
\>[B]{}\Varid{splitCharacterPrefix}\mathbin{::}{}\<[26]%
\>[26]{}\Conid{TextualMonoid}\;\Varid{t}\Rightarrow {}\<[E]%
\\
\>[26]{}\Varid{t}\to \Conid{Maybe}\;(\Conid{Char},\Varid{t}){}\<[E]%
\ColumnHook
\end{hscode}\resethooks
    \end{block}
\end{frame}

\begin{frame}[fragile]
  \frametitle{Парсер как стек монад}
  \begin{block}{Обновленный тип Parser}
    \begin{hscode}\SaveRestoreHook
\column{B}{@{}>{\hspre}l<{\hspost}@{}}%
\column{10}{@{}>{\hspre}l<{\hspost}@{}}%
\column{18}{@{}>{\hspre}l<{\hspost}@{}}%
\column{E}{@{}>{\hspre}l<{\hspost}@{}}%
\>[B]{}\mathbf{newtype}\;{}\<[10]%
\>[10]{}\Conid{Parser}\;\Varid{t}\;\Varid{a}\mathrel{=}{}\<[E]%
\\
\>[10]{}\Conid{Parser}\;{}\<[18]%
\>[18]{}(\Conid{StateT}\;(\Conid{ParserState}\;\Varid{t}){}\<[E]%
\\
\>[18]{}(\Conid{Either}\;(\Conid{ErrorReport}\;\Varid{t}))\;\Varid{a}){}\<[E]%
\ColumnHook
\end{hscode}\resethooks
  \end{block}
\end{frame}

\begin{frame}[fragile]
  \frametitle{Extensible Effects~[ExtEff13]}
  \begin{block}{Тип Eff}
    \begin{hscode}\SaveRestoreHook
\column{B}{@{}>{\hspre}l<{\hspost}@{}}%
\column{E}{@{}>{\hspre}l<{\hspost}@{}}%
\>[B]{}\mathbf{type}\;\Conid{Eff}\;\Varid{r}\;\Varid{a}\mathrel{=}\Conid{Free}\;(\Conid{Union}\;\Varid{r})\;\Varid{a}{}\<[E]%
\ColumnHook
\end{hscode}\resethooks
  \end{block}
  \begin{block}{Пример статического набора эффектов}
    \begin{center}
Eff (Reader Int :> Reader Bool :> Void) a
    \end{center}
  \end{block}
\end{frame}

\begin{frame}[fragile]
  \frametitle{Парсеры на Extensible Effects~[ExtEff13]}

  \begin{block}{Парсер-предикат и его эффекты}
    \begin{hscode}\SaveRestoreHook
\column{B}{@{}>{\hspre}l<{\hspost}@{}}%
\column{9}{@{}>{\hspre}l<{\hspost}@{}}%
\column{15}{@{}>{\hspre}l<{\hspost}@{}}%
\column{E}{@{}>{\hspre}l<{\hspost}@{}}%
\>[B]{}\Varid{sat}\mathbin{::}{}\<[9]%
\>[9]{}(\Conid{Member}\;\Conid{Fail}\;\Varid{r}{}\<[E]%
\\
\>[9]{},\Conid{Member}\;(\Conid{State}\;\Conid{String})\;\Varid{r}{}\<[E]%
\\
\>[9]{})\Rightarrow (\Conid{Char}\to \Conid{Bool})\to \Conid{Eff}\;\Varid{r}\;\Conid{Char}{}\<[E]%
\\
\>[B]{}\Varid{sat}\;\Varid{p}{}\<[9]%
\>[9]{}\mathrel{=}\mathbf{do}\;{}\<[15]%
\>[15]{}\Varid{x}\leftarrow \Varid{item}{}\<[E]%
\\
\>[15]{}\mathbf{if}\;\Varid{p}\;\Varid{x}\;\mathbf{then}\;\Varid{return}\;\Varid{x}\;\mathbf{else}\;\Varid{die}{}\<[E]%
\ColumnHook
\end{hscode}\resethooks
  \end{block}

  \begin{block}{Запуск парсера}
    \begin{hscode}\SaveRestoreHook
\column{B}{@{}>{\hspre}l<{\hspost}@{}}%
\column{E}{@{}>{\hspre}l<{\hspost}@{}}%
\>[B]{}\Varid{parse}\;\Varid{p}\;\Varid{inp}\mathrel{=}\Varid{run}\mathbin{\circ}\Varid{runFail}\mathbin{\circ}\Varid{runState}\;\Varid{inp}\mathbin{\$}\Varid{p}{}\<[E]%
\ColumnHook
\end{hscode}\resethooks
  \end{block}
\end{frame}

\begin{frame}[fragile]
  \frametitle{Парсеры на Extensible Effects~[ExtEff13]}

  \begin{block}{Парсер для слов и его эффекты}
    \begin{hscode}\SaveRestoreHook
\column{B}{@{}>{\hspre}l<{\hspost}@{}}%
\column{10}{@{}>{\hspre}l<{\hspost}@{}}%
\column{E}{@{}>{\hspre}l<{\hspost}@{}}%
\>[B]{}\Varid{word}\mathbin{::}{}\<[10]%
\>[10]{}(\Conid{Member}\;\Conid{Fail}\;\Varid{r}{}\<[E]%
\\
\>[10]{},\Conid{Member}\;(\Conid{State}\;\Conid{String})\;\Varid{r}{}\<[E]%
\\
\>[10]{},\Conid{Member}\;(\Conid{Choose})\;\Varid{r})\Rightarrow \Conid{Eff}\;\Varid{r}\;\Conid{String}{}\<[E]%
\\
\>[B]{}\Varid{word}\mathrel{=}\Varid{some}\;\Varid{letter}{}\<[E]%
\ColumnHook
\end{hscode}\resethooks
  \end{block}

  \begin{block}{Запуск парсера}
    \begin{hscode}\SaveRestoreHook
\column{B}{@{}>{\hspre}l<{\hspost}@{}}%
\column{3}{@{}>{\hspre}l<{\hspost}@{}}%
\column{E}{@{}>{\hspre}l<{\hspost}@{}}%
\>[B]{}\Varid{parseWithChoose}\;\Varid{p}\;\Varid{inp}\mathrel{=}{}\<[E]%
\\
\>[B]{}\hsindent{3}{}\<[3]%
\>[3]{}\Varid{run}\mathbin{\circ}\Varid{runChoice}\mathbin{\circ}\Varid{runFail}\mathbin{\circ}\Varid{runState}\;\Varid{inp}\mathbin{\$}\Varid{p}{}\<[E]%
\ColumnHook
\end{hscode}\resethooks
  \end{block}
\end{frame}

\begin{frame}[fragile]
  \frametitle{Язык Markdown}
  \vspace{0.5cm}
  \screenshotw{11cm}{md-html.png}
\end{frame}

\begin{frame}[fragile]
  \frametitle{AST для Markdown в Haskell}
  \begin{block}{Документ}
    \begin{hscode}\SaveRestoreHook
\column{B}{@{}>{\hspre}l<{\hspost}@{}}%
\column{E}{@{}>{\hspre}l<{\hspost}@{}}%
\>[B]{}\mathbf{type}\;\Conid{Document}\mathrel{=}[\mskip1.5mu \Conid{Block}\mskip1.5mu]{}\<[E]%
\ColumnHook
\end{hscode}\resethooks
  \end{block}
  \begin{block}{Блок}
    \begin{hscode}\SaveRestoreHook
\column{B}{@{}>{\hspre}l<{\hspost}@{}}%
\column{7}{@{}>{\hspre}l<{\hspost}@{}}%
\column{14}{@{}>{\hspre}l<{\hspost}@{}}%
\column{E}{@{}>{\hspre}l<{\hspost}@{}}%
\>[B]{}\mathbf{data}\;{}\<[7]%
\>[7]{}\Conid{Block}{}\<[14]%
\>[14]{}\mathrel{=}\Conid{Blank}{}\<[E]%
\\
\>[14]{}\mid \Conid{Header}\;(\Conid{Int},\Conid{Line}){}\<[E]%
\\
\>[14]{}\mid \Conid{Paragraph}\;[\mskip1.5mu \Conid{Line}\mskip1.5mu]{}\<[E]%
\\
\>[14]{}\mid \Conid{UnorderedList}\;[\mskip1.5mu \Conid{Line}\mskip1.5mu]{}\<[E]%
\\
\>[14]{}\mid \Conid{BlockQuote}\;[\mskip1.5mu \Conid{Line}\mskip1.5mu]{}\<[E]%
\ColumnHook
\end{hscode}\resethooks
  \end{block}
\end{frame}

\begin{frame}[fragile]
  \frametitle{AST для Markdown в Haskell}
  \begin{block}{Строка}
    \begin{hscode}\SaveRestoreHook
\column{B}{@{}>{\hspre}l<{\hspost}@{}}%
\column{E}{@{}>{\hspre}l<{\hspost}@{}}%
\>[B]{}\mathbf{data}\;\Conid{Line}\mathrel{=}\Conid{Empty}\mid \Conid{NonEmpty}\;[\mskip1.5mu \Conid{Inline}\mskip1.5mu]{}\<[E]%
\ColumnHook
\end{hscode}\resethooks
  \end{block}
  \begin{block}{Элементы строки}
    \begin{hscode}\SaveRestoreHook
\column{B}{@{}>{\hspre}l<{\hspost}@{}}%
\column{7}{@{}>{\hspre}l<{\hspost}@{}}%
\column{15}{@{}>{\hspre}c<{\hspost}@{}}%
\column{15E}{@{}l@{}}%
\column{18}{@{}>{\hspre}l<{\hspost}@{}}%
\column{E}{@{}>{\hspre}l<{\hspost}@{}}%
\>[B]{}\mathbf{data}\;{}\<[7]%
\>[7]{}\Conid{Inline}{}\<[15]%
\>[15]{}\mathrel{=}{}\<[15E]%
\>[18]{}\Conid{Plain}\;\Conid{String}{}\<[E]%
\\
\>[15]{}\mid {}\<[15E]%
\>[18]{}\Conid{Bold}\;\Conid{String}{}\<[E]%
\\
\>[15]{}\mid {}\<[15E]%
\>[18]{}\Conid{Italic}\;\Conid{String}{}\<[E]%
\\
\>[15]{}\mid {}\<[15E]%
\>[18]{}\Conid{Monospace}\;\Conid{String}{}\<[E]%
\ColumnHook
\end{hscode}\resethooks
  \end{block}
\end{frame}

\begin{frame}[fragile]
  \frametitle{Конструирование AST}
    \begin{block}{Парсер для документа}
      \vspace{-1.4em}
      \begin{hscode}\SaveRestoreHook
\column{B}{@{}>{\hspre}l<{\hspost}@{}}%
\column{9}{@{}>{\hspre}l<{\hspost}@{}}%
\column{24}{@{}>{\hspre}l<{\hspost}@{}}%
\column{E}{@{}>{\hspre}l<{\hspost}@{}}%
\>[B]{}\Varid{doc}\mathbin{::}{}\<[9]%
\>[9]{}\Conid{\Conid{TM}.TextualMonoid}\;\Varid{t}\Rightarrow \Conid{Parser}\;\Varid{t}\;\Conid{Document}{}\<[E]%
\\
\>[B]{}\Varid{doc}\mathrel{=}{}\<[9]%
\>[9]{}\Varid{many}\;\Varid{block}{}\<[E]%
\\
\>[9]{}\mathbf{where}\;\Varid{block}\mathrel{=}{}\<[24]%
\>[24]{}\Varid{blank}\mathbin{\raisebox{0.3ex}{$\scriptstyle < | >$}}\Varid{header}\mathbin{\raisebox{0.3ex}{$\scriptstyle < | >$}}\Varid{paragraph}{}\<[E]%
\\
\>[24]{}\mathbin{\raisebox{0.3ex}{$\scriptstyle < | >$}}\Varid{unorderdList}\mathbin{\raisebox{0.3ex}{$\scriptstyle < | >$}}\Varid{blockquote}{}\<[E]%
\\
\>[24]{}\mathbin{\raisebox{0.3ex}{$\scriptstyle < | >$}}\Varid{blockMath}{}\<[E]%
\ColumnHook
\end{hscode}\resethooks
      \vspace{-1.45em}
    \end{block}
    \begin{block}{Пример парсера для заголовка}
      \vspace{-1.4em}
      \begin{hscode}\SaveRestoreHook
\column{B}{@{}>{\hspre}l<{\hspost}@{}}%
\column{11}{@{}>{\hspre}l<{\hspost}@{}}%
\column{19}{@{}>{\hspre}l<{\hspost}@{}}%
\column{E}{@{}>{\hspre}l<{\hspost}@{}}%
\>[B]{}\Varid{header}\mathbin{::}\Conid{\Conid{TM}.TextualMonoid}\;\Varid{t}\Rightarrow \Conid{Parser}\;\Varid{t}\;\Conid{Block}{}\<[E]%
\\
\>[B]{}\Varid{header}\mathrel{=}{}\<[11]%
\>[11]{}\mathbf{do}{}\<[E]%
\\
\>[11]{}\Varid{hashes}{}\<[19]%
\>[19]{}\leftarrow \Varid{token}\mathbin{\$}\Varid{some}\mathbin{\$}\Varid{char}\;\text{\tt '\#'}{}\<[E]%
\\
\>[11]{}\Varid{text}{}\<[19]%
\>[19]{}\leftarrow \Varid{nonEmptyLine}{}\<[E]%
\\
\>[11]{}\Varid{return}\mathbin{\$}\Conid{Header}\;(\Varid{length}\;\Varid{hashes},\Varid{text}){}\<[E]%
\ColumnHook
\end{hscode}\resethooks
    \end{block}
\end{frame}

\begin{frame}[fragile]
  \frametitle{Результаты}
  \begin{enumerate}
    \item Разработана библиотека монадических парсеров, полиморфных по входным данным. Акцент сделан на подробность сообщений об ошибках. 
    \item Разработан транслятор подмножества Markdown с \LaTeX-вставками в HTML.
    \item Начата разработка библиотеки парсеров, основанных на Extensible Effects.
    \item Исходный код доступен в Git-репозиториях: \\
        \small{\url{https://github.com/geo2a/markdown_monparsing}} \\
        \small{\url{https://github.com/geo2a/ext-effects-parsers}}
  \end{enumerate}
\end{frame}

\end{document}
